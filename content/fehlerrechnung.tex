\newpage
\section{Fehlerrechnung}
\label{sec:fehlerrechnung}

\subsection{Mittelwert}
\label{subsec:mittelwert}
  Im folgenden werden die Messwerte $x_i$ bei einer
  Anzahl von $n$ Messungen betrachtet.\\
  \\
  Die Formel zur Berechnung des Mittelwertes ist:

  \begin{equation}
    \bar{x}=\frac{1}{n} \sum_{i=1}^n x_i \quad .
    \label{eqn:mittelwert}
  \end{equation}
  \\\\
  Die dazugehörige Standardabweichung lautet:

  \begin{equation}
    \sigma = \sqrt{\frac{1}{n-1} \sum_{i=1}^n (x_i - \overline{x})^2} \quad .
    \label{eqn:std1}
  \end{equation}
  \\\\
  Wenn nur Stichproben einer Messung ausgewertet werden,
  wird die Standardabweichung angepasst zu:

  \begin{equation}
    \sigma_s = \sqrt{\frac{1}{n\,(n-1)} \sum_{i=1}^n (x_i - \overline{x})^2} \quad .
    \label{eqn:std2}
  \end{equation}
  \\\\

\subsection{Fehlerfortpflanzung}
\label{subsec:fehlerfortplanzung}
  Sind mehrere Messwerte durch eine Funktion $y=y(x_1, x_2, \dots, x_n)$ abhängig,
  muss dies beachtet werden. \\\\
  Der wahrscheinlichste Fehler ist der sogenannte Gauß'sche Fehler.
  Für unabhängige $x_n$ ist dieser gegeben durch:\\

  \begin{equation}
    \increment y=\sqrt{
    \left(\dfrac{\partial y}{\partial {x_1}}\cdot\increment x_1\right)^{\!2} +
    \left(\dfrac{\partial y}{\partial {x_2}}\cdot\increment x_2\right)^{\!2}
    +\,\dotsb\,+
    \left(\dfrac{\partial y}{\partial {x_n}}\cdot\increment x_n\right)^{\!2}
    }
    \label{eqn:gaußfehler}
  \end{equation}
