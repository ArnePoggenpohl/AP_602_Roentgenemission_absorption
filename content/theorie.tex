\section{Theorie}
\label{sec:Theorie}
Röntgenstrahlung ensteht durch abgebremste Elektronen, die auf ein Anodenmaterial
treffen, welches in Abhängigkeit der Energien der Elektronen Röntgenstrahlung
emittiert. Das Spektrum der Röntgenstrahlung besteht aus einem kontinuierlichen Bremsspektrum
und für das Anodenmaterial, in diesem Fall Kupfer, charakteristische Peaks.
Diese sind bei genau den Wellenlängen zu finden, die den Sprüngen zwischen den
Elektronenbahnen entsprechen.\\
Das Bremsspektrum entsteht durch die Abbremsung der Elektronen im Coulombfeld, daher ist
dies ein kontinuierliches Spektrum mit der minimalen Wellenlänge
\begin{equation}
  \lambda_\symup{min} = \dfrac{hc}{e_\symup{0}U} \quad,
  \label{eqn:lambda_min}\\
\end{equation}
die genau der Energie der beschleunigten Elektronen entspricht.\\
Das charakteristische Spektrum entsteht durch Quantensprünge der Elektronen zwischen
den Schalen der Kupferatome. Es gilt
\begin{equation}
  h \nu = E_\symup{m} - E_\symup{n}
  \label{eqn:hnu}\\
\end{equation}
für die Energie der emittierten Photonen bei den Quantensprüngen, dabei ist $E_\symup{n}$ die
Bindungsenergie des Elektron auf der n-ten Schale und gegeben durch
\begin{equation}
  E_\symup{n} = -R_\symup{\infty} z_\symup{eff}^2 \dfrac{1}{n^2}
  \label{eqn:E_n}\\
\end{equation}
mit der Rydbergenergie $R_\symup{\infty} = \SI{13.6}{\electronvolt}$ und der effektiven
Kernladungszahl $z_\symup{eff} = z - \sigma$, wobei $\sigma$ die Abschirmkonstante ist, welche
in diesem Versuch empirisch bestimmt wird.\\\\
Die charakteristischen Peaks der Röntgenstrahlung werden $K_\symup{\alpha},K_\symup{\beta},
L_\symup{\alpha}\dots$ Linien genannt, dabei steht das $K$ für die jeweilige Schale auf der die Übergänge
enden und das $\symup{\alpha}$ für die Schale von der das Elektron stammt. Die Energie
der $K_\symup{\alpha}$ Linie ist dann gegeben durch:
\begin{align}
  E_{K_\symup{\alpha}} = R_\symup{\infty} \left(z - \sigma_\symup{1} \right)^2\dfrac{1}{1^2} -
  R_\symup{\infty}\left(z - \sigma_\symup{2}\right)^2 \dfrac{1}{2^2}
  \label{eqn:E_K_a}
\end{align}
\begin{align}
  E_{K_\symup{\beta}} = R_\symup{\infty} \left(z - \sigma_\symup{1} \right)^2\dfrac{1}{1^2} -
  R_\symup{\infty}\left(z - \sigma_\symup{3}\right)^2 \dfrac{1}{3^2}
  \label{eqn:E_K_b}
\end{align}
\newpage
Der zweite Teil des Versuchs beschäftigt sich mit der Absorption von Röntgenstrahlung durch
verschiedene Materialien. Dabei spielen vor allem der Compton- und der Photoeffekt eine
tragende Rolle für Energien unter $\SI{1}{\mega\electronvolt}$.\\
Der Absorptionskoeffizient fällt mit zunehmender Energie ab springt jedoch immer dann
auf ein höheres Niveau, wenn die Energien gerade größer sind als die Bindungsenergie der
nächsten inneren Schale. Die Absorptionskanten $h\nu_\symup{abs} = E_\symup{n} - E_\symup{\infty}$
liegen nahezu bei den Bindungsenergien der Elektronen. Es kann jeweils nur eine $K$ -Kante
beobachtet werden, aufgrund der sogennannten Feinstruktur jedoch drei $L$ -Kanten,
welche als $L_\symup{I},L_\symup{II} \text{ und } L_\symup{III}$
bezeichnet werden. Unter Berücksichtigung der Feinstruktur kann mithilfer der
Sommerfeldschen Feinstrukturformel die Bindungsenergie $E_\symup{n,j}$ der Elektronen
berechnet werden.
\begin{equation}
  E_\symup{n,j} = -R_\symup{\infty} \left(z_\symup{eff,1}^2 \cdot \dfrac{1}{n^2}
  + \alpha^2 z_\symup{eff,2}^4 \cdot \dfrac{1}{n^3}\left(\dfrac{1}{j+\frac{1}{2}}-
  \dfrac{3}{4n}\right)\right)
  \label{eqn:E_n,j}\\
\end{equation}
Dabei ist $\alpha$ die Sommerfeldsche Feinstrukturkonstante, $n$ die Hauptquantenzahl und
$j$ der Gesamtdrehimpuls des betrachteten Elektrons. Da die Abschirmkonstante $\sigma_\symup{L}$
kompliziert zu bestimmen ist, wird im Versuch eine Vereinfachung zur Hilfe genommen,
bei der die Energiedifferenz der $L_\symup{II}$ und $L_\symup{III}$ -Linien betrachtet wird.
\begin{equation}
  \sigma_\symup{L} = Z - \left(\dfrac{4}{\alpha}\sqrt{\dfrac{\symup{\Delta} E_\symup{L}}{R_\symup{\infty}}}
  - \dfrac{5 \symup{\Delta} E_\symup{L} }{R_\symup{\infty}} \right)^{\sfrac{1}{2}}
  \left(1 + \dfrac{19}{32} \alpha^2 \dfrac{\symup{\Delta} E_\symup{L}}{R_\symup{\infty}} \right)^{\sfrac{1}{2}}
  \label{eqn:sigma_L}\\
\end{equation}
Dabei ist $Z$ die Ordnungszahl und $\symup{\Delta} E_\symup{L} = E_{L_\symup{II}} -E_{L_\symup{III}}$.\\
Im Versuch werden die Spektren jeweils mithilfe der Bragg-Reflektion aufgenommen. Mithilfe
der Bragg-Bedingung

\begin{equation}
  2 d \sin{\Theta} = n \lambda
  \label{eqn:bragg}
\end{equation}

lassen sich die gemessen Winkel den zughörigen Wellenlängen zuordnen. Dabei ist $d$ die
Gitterkonstante. In diesem Fall ist $d = \SI{201.4}{\pico\meter}$ für den hier verwendeten
$LiF$ -Kristall. Über den Zusammenhang $E=h\nu$ lässt sich diese Wellenlänge in eine
Energie umrechnen.

\begin{equation}
  E = \dfrac{hc}{2d\sin{\Theta}}
  \label{eqn:E}
\end{equation}
