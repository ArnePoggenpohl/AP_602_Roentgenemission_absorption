\section{Theorie}
\label{sec:Theorie}
Röntgenstrahlung ensteht durch Beschleunigte Elektronen, die auf ein Anodenmaterial
treffen, welches, in Abhängigkeit der Energien der Elektronen, Röntgenstrahlung
emittiert. Das Spektrum der Röntgenstrahlug besteht aus einem kontinuierlichen Bremsspektrum
und für das Anodenmaterial, in diesem Fall Kupfer, charakteristische Peaks.
Diese sind bei genau den Wellenlängen zu finden, die den Sprüngen zwischen den
Elektronenbahnen entsprechen.\\
Das Bremsspektrum entsteht durch die Abbremsung der Elektronen im Coulombfeld, daher ist
dies ein kontinuierliches Spektrum mit der minimalen Wellenlänge
\begin{equation}
  \lambda_\symup{min} = \dfrac{hc}{e_\symup{0}U} \quad,
  \label{eqn:lambda_min}\\
\end{equation}
die außerdem genau der Energie der beschleunigten Elektronen entspricht.\\
Das charakteristische Spektrum entsteht durch Quantensprünge der Elektronen zwischen
den Schalen der Kupfer Atome. Es gilt
\begin{equation}
  h \nu = E_\symup{m} - E_\symup{n}
  \label{eqn:hnu}\\
\end{equation}
für die Energie der emittierten Photonen bei den Quantensprüngen, dabei ist $E_\symup{n}$ die
Bindungsenergie des Elektron auf der n-ten Schale und gegeben durch
\begin{equation}
  E_\symup{n} = -R_\symup{\infty} z_\symup{eff}^2 \dfrac{1}{n^2}
  \label{eqn:E_n}\\
\end{equation}
mit der Rydbergenergie $R_\symup{\infty} = \SI{13.6}{\electronvolt}$ und der effektiven
Kernladungszahl $z_\symup{eff} = z - \sigma$, wobei $\sigma$ die Abschirmkonstante ist, welche
in diesem Versuch empirisch bestimmt wird.
