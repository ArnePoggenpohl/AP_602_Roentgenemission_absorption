\section{Diskussion}
\label{sec:Diskussion}

Für den ersten Versuchsteil lässt sich erkennen, dass das Maximum des Spektrums
bei $\Theta = \SI{28.2}{\degree}$ leigt. Dieser Wert war zu erwarten und anhand der
Messkurve lässt sich außerdem die Bragg-Bedingung bestätigen.\\
Der zweite Versuchsteil lieferte folgende Ergebnisse für die $K_\symup{\alpha}$ und
die $K_\symup{\beta}$ Linien:\\
\begin{align*}
  E_\alpha &= \SI{8.08}{\kilo\eV}\\
  E_{\alpha,\symup{Lit}} &= \SI{8.053}{\kilo\eV}\\
  \text{Abweichung} &= \SI{0.3}{\percent}\\\\
  E_\beta &= \SI{9.01}{\kilo\eV}\\
  E_{\beta,\symup{Lit}} &= \SI{8.912}{\kilo\eV}\\
  \text{Abweichung} &= \SI{1.1}{\percent}\\
\end{align*}
Daraus wurde die Abschirmkonstante
\begin{align*}
  \sigma_K = \num{20.73}
\end{align*}
berechnet, welche im Vergleich zu den später errechneten Werten deutlich höher ist. Die
Halbwertsbreite der $K_\beta$ Linie beträgt $\SI{0.4}{\degree}$ und die der
$K_\alpha$ Linie $\SI{0.35}{\degree}$. Da die Abtastrate in $\SI{0.2}{\degree}$-Schritten erfolgt
ist, sind diese Werte für die Halbwertesbreite ungenau. Durch die groben Schritte ist es möglich,
dass der
$K_{\alpha/\beta}$ Winkel nicht genau getroffen wurde und somit die Maxima zu niedrig sind.\\
Auch die Ergebnisse für diesen Versuchsteil liegen in den erwarteten Bereichen,
was für eine saubere Messung spricht.\\
Für den letzten Versuchsteil erwies sich die Auswertung als sehr umständlich, da
das erkennen der $K$ und $L$ -Linien in den Graphen nicht genau möglich war. Deswegen
ist auch eine Beurteilung der Ergebnisse sehr schwierig. Für die einzelnen
Elemente Brom, Strontium, Zirkonium und Quecksilber ergaben sich folgende
Abschirmzahlen:
\begin{align*}
  \sigma_\symup{K,Br} &= \num{2.98}\\
  \sigma_\symup{K,Sr} &= \num{4.00}\\
  \sigma_\symup{K,Zr} &= \num{4.58}\\
  \sigma_\symup{L,Hg} &= \num{3.34}\\
\end{align*}

\nocite{sample}
\nocite{alpha}
