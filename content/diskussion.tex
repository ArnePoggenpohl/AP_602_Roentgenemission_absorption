\section{Diskussion}
\label{sec:Diskussion}

Für den ersten Versuchsteil lässt sich erkennen, dass das Maximum des Spektrums
bei $2 \Theta = \SI{28.2}{\degree}$ leigt. Dieser Wert war zu erwarten und anhand der
Messkurve lässt sich außerdem die Bragg-Bedingung bestätigen.\\
Der zweite Versuchsteil lieferte folgende Ergebnisse für die $K_\symup{\alpha}$ und
die $K_\symup{\beta}$ Linien. Die Literaturwerte stammen von der Website \cite{alpha}.\\
\begin{align*}
  E_\alpha &= \SI{8.08}{\kilo\eV}\\
  E_{\alpha,\symup{Lit}} &= \SI{8.053}{\kilo\eV}\\
  \text{Abweichung} &= \SI{0.3}{\percent}\\\\
  E_\beta &= \SI{9.01}{\kilo\eV}\\
  E_{\beta,\symup{Lit}} &= \SI{8.912}{\kilo\eV}\\
  \text{Abweichung} &= \SI{1.1}{\percent}\\
\end{align*}
Daraus wurden die Abschirmkonstanten
\begin{align*}
  \sigma_\symup{1} &= \num{3.26} \\
  \sigma_\symup{2} &= \num{12.5} \\
  \sigma_\symup{3} &= \num{29}
\end{align*}
berechnet, welche im Vergleich zu den später errechneten Werten teils deutlich höher sind. \\
Die Halbwertsbreite der $K_\beta$ Linie beträgt $\SI{0.4}{\degree}$ und die der
$K_\alpha$ Linie $\SI{0.35}{\degree}$. Da die Abtastrate in $\SI{0.2}{\degree}$-Schritten erfolgt
ist, sind diese Werte für die Halbwertesbreite ungenau. Durch die groben Schritte ist es möglich,
dass der
$K_{\alpha/\beta}$ Winkel nicht genau getroffen wurde und somit die Maxima zu niedrig sind.\\
Auch die Ergebnisse für diesen Versuchsteil liegen in den erwarteten Bereichen,
was für eine saubere Messung spricht.\\
Für den letzten Versuchsteil erwies sich die Auswertung als sehr umständlich, da
das erkennen der $K$ und $L$ -Linien in den Graphen nicht genau möglich war. Deswegen
ist auch eine Beurteilung der Ergebnisse schwierig. Für die einzelnen
Elemente Brom, Strontium, Zirkonium und Quecksilber ergaben sich folgende
Abschirmzahlen. Die Literaturwerte kommen von der Seite \cite{k_edge}.

\begin{table}[H]
  \centering
  \caption{Abschirmzahlen im Vergleich zum Literaturwert.}
    \begin{tabular}{c c c c}
    \toprule
    Element & $\sigma$ & $\sigma_\symup{Lit}$ & Abweichung\\
    \midrule
    Brom & 3,51 & 3,52 & -0,3\% \\
    Strontium & 4,00 & 3,59 & +10,3\% \\
    Zirkonium & 4,58 & 3,62 & +21,0\% \\
    Quecksilber & 3,34 & 3.58 & -7,2\% \\
    \bottomrule
  \end{tabular}
  \label{tab:sigma_lit}
\end{table}

Die Berechnung der Rydbergkonstante lieferte den Wert:
\begin{align*}
  R = \SI{8.47(76)}{\eV}
\end{align*}
Dieser Wert ist ca. $\SI{60}{\percent}$ kleiner als der in der Versuchanleitung angegebener
Wert. Dies ist auch schon daran zu erkennen, dass die Abweichung der $\sigma_\symup{K}$'s
von Brom bis zum Zirkonium immer weiter ins positive Ansteigen. Wenn sie das nicht tun
würden, dann wäre die Rydbergkonstante auch größer.
\nocite{sample}
\nocite{alpha}
