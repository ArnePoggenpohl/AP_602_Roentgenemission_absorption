\newpage
\section{Auswertung}
\label{sec:Auswertung}

\subsection{Überprüfung der Bragg-Bedingung}
Zur Überprüfung der Bragg-Bedingung wird an dem Gerät ein fester Kristallwinkel von
$\theta = \SI{14}{\degree}$ eingestellt. Es wird nun ein Maximum bei $\SI{28}{\degree}$
erwartet. Um dies festzustellen wird mit einem Geiger-Müller-Zählrohr im Bereich von
$\SI{26}{\degree}$ bis $\SI{30}{\degree}$ in $\SI{0.1}{\degree}$ Schritten die Intensität
der Röntgenstrahlung gemessen.

\begin{figure}[H]
  \centering
  \includegraphics{plot_bragg.pdf}
  \caption{Graph zur Überprüfung der Bragg-Gleichung.}
  \label{fig:plot_bragg}
\end{figure}

Es ist zu sehen, dass das Maximum bei $\SI{28.2}{\degree}$ liegt und somit leicht verschoben
ist. Allerdings wird durch den gesamten Verlauf der Kurve deutlich, dass die Bragg-Bedinung
erfüllt ist.


\subsection{Das Emissionsspektrum einer Cu-Röntgenröhre}

Um das Emissionsspektrum der Kupferröntgenröhre zu bestimmen wird das Röntgenspektrum
im Bereich von $\SI{4}{\degree}$ bis $\SI{26}{\degree}$ gemessen.

\begin{figure}[H]
  \centering
  \includegraphics{plot_emission.pdf}
  \caption{Emissionsspektrum einer Cu-Röntgenröhre.}
  \label{fig:plot_emission}
\end{figure}

Die $K_\beta$ Linie liegt bei $\SI{20.1}{\degree}$ und die $K_\alpha$ Linie liegt bei $\SI{22.3}{\degree}$.
Anhand der Messdaten wird die Halbwertsbreite der $K_\alpha$ und $K_\beta$ Linie bestimmt. \\
Für die $K_\beta$ Linie beträgt die Halbwertsbreite $\SI{0.8}{\degree}$ und für die
$K_\alpha$ Linie $\SI{0.7}{\degree}$. Da die Abtastrate in $\SI{0.2}{\degree}$-Schritten erfolgt
ist, sind diese Angaben ungenau. Durch die groben Schritte ist es möglich, dass der
$K_{\alpha/\beta}$ Winkel nicht genau getroffen wurde und somit die Maxima zu niedrig sind. \\

Mit der Gleichung ??? und ??? lässt sich die Abschirmkonstante $\sigma_L$ bestimmten.

\begin{align*}
  \sigma_L = \num{29.0}
\end{align*}

\subsection{Das Absorptionsspektrum}

In diesem Bereich wird die Abschirmzahl $\sigma_K$ aus dem Absorptionswinkel bestimmt.
Der benötigte Winkel für die Bragg-Bedingung wird dem jeweiligen Graph der Absorptionskurve
entnommen. \\

\begin{figure}[H]
  \centering
  \includegraphics{plot_abs_br.pdf}
  \caption{Absorptionsspektrum mit einem Bromabsorber.}
  \label{fig:plot_br}
\end{figure}

Der Winkel für die K-Kante bei dem Bromabsorber \eqref{fig:plot_br} liegt bei
\begin{align*}
  \theta_\symup{K,Br} = \SI{13.2}{\degree}.
\end{align*}

Daraus ergibt sich mit Gleichung ???(E=hv) und ???(Bragg) die Energie
\begin{align*}
  E_\symup{K,Br} = \SI{13.49}{\kilo\eV}.
\end{align*}

Unter Ausnutzung der Gleichung ???(K-alpha), in welcher $\sigma_2 = 0$ ist, da es sich um
die K-Schale handelt, errechnet sich die Abschirmzahl zu
\begin{align*}
  \sigma_\symup{K,Br} = .
\end{align*}

\begin{figure}[H]
  \centering
  \includegraphics{plot_abs_sr.pdf}
  \caption{Absorptionsspektrum mit einem Strontiumabsorber.}
  \label{fig:plot_sr}
\end{figure}

\begin{figure}[H]
  \centering
  \includegraphics{plot_abs_zr.pdf}
  \caption{Absorptionsspektrum mit einem Zirkoniumabsorber.}
  \label{fig:plot_zr}
\end{figure}

Analog werden die Abschirmzahlen für einen Strontiumabsorber \eqref{fig:plot_sr} und einen
Zirkoniumabsorber \eqref{fig:plot_zr} berechnet.

\begin{align*}
  \theta_\symup{K,Sr} &= \SI{11.3}{\degree}\\
  E_\symup{K,Sr} &= \SI{15.72}{\kilo\eV}\\
  \sigma_\symup{K,Sr} &=\\
\end{align*}
\begin{align*}
  \theta_\symup{K,Zr} &= \SI{10.4}{\degree}\\
  E_\symup{K,Zr} &= \SI{17.06}{\kilo\eV}\\
  \sigma_\symup{K,Zr} &=\\
\end{align*}


\begin{figure}[H]
  \centering
  \includegraphics{plot_abs_hg.pdf}
  \caption{Absorptionsspektrum mit einem Quecksilberabsorber.}
  \label{fig:plot_hg}
\end{figure}

Die Abschirmkonstante für das Quecksilber wird auf eine andere Weise bestimmt. Zunächst
werden die $L_{II}$ und $L_{III}$ Kanten aus den Messwerten \eqref{fig:plot_hg} bestimmt.
Diese werden mithilfe der Gleichung ???(E mit Bragg) in eine Energie überführt und danach
mit Gleichung ???(sigma-L) in die Abschirmzahl überführt.

\begin{align*}
  L_{II} &= \SI{7.6}{\degree}\\
  L_{III}&= \SI{8.3}{\degree}\\
  \Delta E &= E_{L_{III}}-E_{L_{III}} = \SI{1.95}{\kilo\eV}\\\\
  \Rightarrow \sigma_\symup{L,Hg} &= \num{3.34}\\
\end{align*}
