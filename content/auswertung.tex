\newpage
\section{Auswertung}
\label{sec:Auswertung}

\subsection{Überprüfung der Bragg-Bedingung}
Zur Überprüfung der Bragg-Bedingung wird an dem Gerät ein fester Kristallwinkel von
$\theta = \SI{14}{\degree}$ eingestellt. Es wird nun ein Maximum bei $\SI{28}{\degree}$
erwartet. Um dies festzustellen wird mit einem Geiger-Müller-Zählrohr im Bereich von
$\SI{26}{\degree}$ bis $\SI{30}{\degree}$ in $\SI{0.1}{\degree}$ Schritten die Intensität
der Röntgenstrahlung gemessen.

\begin{figure}[H]
  \centering
  \includegraphics{plot_bragg.pdf}
  \caption{Graph zur Überprüfung der Bragg-Gleichung.}
  \label{fig:plot_a2}
\end{figure}

Es ist zu sehen, dass das Maximum bei $\SI{28.2}{\degree}$ liegt und somit leicht verschoben
ist. Allerdings wird durch den gesamten Verlauf der Kurve deutlich, dass die Bragg-Bedinung
erfüllt ist.


\subsection{Das Emissionsspektrum einer Cu-Röntgenröhre}

\begin{figure}[H]
  \centering
  \includegraphics{plot_emission.pdf}
  \caption{Emissionsspektrum einer Cu-Röntgenröhre.}
  \label{fig:plot_a2}
\end{figure}
