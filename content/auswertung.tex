\newpage
\section{Auswertung}
\label{sec:Auswertung}

\subsection{Überprüfung der Bragg-Bedingung}
Zur Überprüfung der Bragg-Bedingung wird an dem Gerät ein fester Kristallwinkel von
$\theta = \SI{14}{\degree}$ eingestellt. Es wird nun ein Maximum bei $\SI{28}{\degree}$
erwartet. Um dies festzustellen wird mit einem Geiger-Müller-Zählrohr im Bereich von
$\SI{26}{\degree}$ bis $\SI{30}{\degree}$ in $\SI{0.1}{\degree}$ Schritten die Intensität
der Röntgenstrahlung gemessen.

\begin{figure}[H]
  \centering
  \includegraphics{plot_bragg.pdf}
  \caption{Graph zur Überprüfung der Bragg-Gleichung.}
  \label{fig:plot_bragg}
\end{figure}

Es ist zu sehen, dass das Maximum bei $\SI{28.2}{\degree}$ liegt und somit leicht verschoben
ist. Allerdings wird durch den gesamten Verlauf der Kurve deutlich, dass die Bragg-Bedinung
erfüllt ist.


\subsection{Das Emissionsspektrum einer Cu-Röntgenröhre}

Um das Emissionsspektrum der Kupferröntgenröhre zu bestimmen wird das Röntgenspektrum
im Bereich von $\SI{4}{\degree}$ bis $\SI{26}{\degree}$ gemessen.

\begin{figure}[H]
  \centering
  \includegraphics{plot_emission.pdf}
  \caption{Emissionsspektrum einer Cu-Röntgenröhre.}
  \label{fig:plot_emission}
\end{figure}

Die $K_\beta$ Linie liegt bei $\SI{20.1}{\degree}$ und die $K_\alpha$ Linie liegt bei $\SI{22.3}{\degree}$.
Anhand der Messdaten wird die Halbwertsbreite der $K_\alpha$ und $K_\beta$ Linie bestimmt. \\
Für die $K_\beta$ Linie beträgt die Halbwertsbreite $\SI{0.8}{\degree}$ und für die
$K_\alpha$ Linie $\SI{0.7}{\degree}$. Da die Abtastrate in $\SI{0.2}{\degree}$-Schritten erfolgt
ist, sind diese Angaben ungenau. Durch die groben Schritte ist es möglich, dass der
$K_{\alpha/\beta}$ Winkel nicht genau getroffen wurde und somit die Maxima zu niedrig sind. \\

Mit der Gleichung ??? und ??? lässt sich die Abschirmkonstante $\sigma_L$ bestimmten.

\begin{align*}
  \sigma_L = 29
\end{align*}

\subsection{Das Absorptionsspektrum}

In diesem Bereich wird die Abschirmzahl $\sigma_K$ aus dem Absorptionswinkel bestimmt.
Der benötigte Winkel für die Bragg-Bedingung wird dem jeweiligen Graph der Absorptionskurve
entnommen. \\
Der Winkel für den Bromabsorber beträgt

\begin{align*}
  \theta_\symup{Br_K} = \SI{13.2}{\degree}.
\end{align*}

Daraus ergibt sich mit Gleichung ???(E=hv) und ???(Bragg) die Energie

\begin{align*}
  E_\symup{Br_K} = \SU{13.49}{\kilo\elektron\volt}.
\end{align*}

Unter Ausnutzung der Gleichung ??? errechnet sich die Abschirmzahl zu

\begin{align*}
  \sigma_\symup{Br_K} = .
\end{align*}

\begin{figure}[H]
  \centering
  \includegraphics{plot_abs_br.pdf}
  \caption{Absorptionsspektrum mit einem Bromabsorber.}
  \label{fig:plot_br}
\end{figure}

\begin{figure}[H]
  \centering
  \includegraphics{plot_abs_sr.pdf}
  \caption{Absorptionsspektrum mit einem Strontiumbsorber.}
  \label{fig:plot_sr}
\end{figure}

\begin{figure}[H]
  \centering
  \includegraphics{plot_abs_zr.pdf}
  \caption{Absorptionsspektrum mit einem Zirkoniumabsorber.}
  \label{fig:plot_zr}
\end{figure}

\begin{figure}[H]
  \centering
  \includegraphics{plot_abs_hg.pdf}
  \caption{Absorptionsspektrum mit einem Quecksilberabsorber.}
  \label{fig:plot_hg}
\end{figure}
